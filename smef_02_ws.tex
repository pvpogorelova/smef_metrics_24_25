\documentclass [12pt,a4paper]{article}%размер листа, размер шрифта и тип документа (статья)

\usepackage[12pt]{extsizes}
\usepackage[utf8]{inputenc}% кодировка для Windows
\usepackage[english,russian]{babel}%русский язык

\usepackage{times}

\usepackage[pdftex]{graphicx}% подключаем пакет графики
\usepackage{color}% подключаем другие пакеты, которые будут использоваться
\usepackage[colorlinks]{hyperref}
\usepackage{amsmath,amssymb}%для фигурных скобок системы
\usepackage{lscape} % поддержка страниц в альбомной ориентации
\usepackage[left=30mm, top=25mm, right=20mm, bottom=20mm, nofoot]{geometry}
\usepackage{setspace}
\usepackage{url}
\usepackage{tempora}
\usepackage{caption} % подписи к рисункам в русской типографской традиции
%чтобы в нумерации была точка
\DeclareCaptionLabelFormat{gosttable}{Table #2}
\DeclareCaptionLabelSeparator{gost}{.~}
\captionsetup{labelsep=gost}
\captionsetup[figure]{labelformat=gostfigure}
\captionsetup[table]{labelformat=gosttable}

\usepackage{graphicx}%Вставка картинок правильная
\usepackage{float}%"Плавающие" картинки
\usepackage{wrapfig}%Обтекание фигур (таблиц, картинок и прочего)

\usepackage{cmap}					
\usepackage{mathtext} 				
\usepackage[T2A]{fontenc}		
\usepackage[utf8]{inputenc}			
\usepackage[english,russian]{babel}	

\usepackage{amsfonts,amssymb,amsthm,mathtools} 
\usepackage{amsmath}
\usepackage{icomma} 


\newcommand*{\hm}[1]{#1\nobreak\discretionary{}
	{\hbox{$\mathsurround=0pt #1$}}{}}

\usepackage{graphicx} 
\graphicspath{{images/1/}{images/2/}{images/4/}}
\setlength\fboxsep{3pt} 
\setlength\fboxrule{1pt} 
\usepackage{wrapfig} 


\usepackage{array,tabularx,tabulary,booktabs} 
\usepackage{longtable}  
\usepackage{multirow} 
\usepackage{array}


\newcolumntype{Y}{>{\centering\arraybackslash}X}

\usepackage{indentfirst}
\usepackage{hyperref}
\usepackage{float}


\usepackage[dvips]{graphicx}

\usepackage{fancyhdr} % весёлые колонтитулы
\pagestyle{fancy}
\lhead{Эконометрика 1 (углубленный курс),  НИУ ВШЭ}
\chead{}
\rhead{Семинары: Погорелова П.В.}

\renewcommand{\headrulewidth}{0.4pt}
\renewcommand{\footrulewidth}{0.4pt}


\onehalfspacing %полуторный интервал

%\footheight=10pt
\footskip=10mm
\graphicspath{{figures/}}
\newcommand{\rouble}{{\rm{Р}\kern-.635em\rule[.5ex]{.52em}{.04em}\kern.11em}}


\begin{document}


\begin{center}
\large Семинар 2.\\
Решение.
\end{center}

\begin{enumerate}

\item Рассмотрим нормальную классическую линейную модель множественной регрессии $y_i = \beta_1 + \beta_2x_{i2} + ... +\beta_kx_{ik} + \varepsilon_i$ с неслучайными регрессорами.
Дополнительно известно, что на самом деле $\beta_2 = ... = \beta_k = 0$.
\begin{itemize}
\item[(a)] Найдите $\mathbb{E}(R^2)$.
\item[(b)] Найдите $\mathbb{E}(R^2_{adj.})$.
\item[(c)] Покажите, что $nR^2 \sim \chi^2(k-1)$.
\end{itemize}

\textbf{Решение:}
\item[(a)] Модель без ограничений: 
\[
Y_i = \beta_1 + \beta_2X_{i2} + ... +\beta_kX_{ik} + \varepsilon_i.
\]

Модель с ограничениями (истинная модель!):

\[
Y_i = \beta_1 + \varepsilon_i.
\]

Тогда F-статистика при справедливости $H_0$ имеет следующий вид:
\[
F = \frac{R^2(k-1)}{(1-R^2)/(n-k)} \sim F(k-1,n-k).
\]

Выразим $R^2$:
\[
R^2(n-k) = F(1-R^2)(n-k).
\]

\textbf{Утверждение №1:} Если $X \sim F(k_1,k_2)$, то $Y = \frac{\frac{k_1}{k_2}X}{1+\frac{k_1}{k_2}X} \sim Beta \left(\frac{k_1}{2}, \frac{k_2}{2}\right)$.

Используя утверждение №1, получаем:

\[
R^2 = \frac{(k-1)F}{(n-k) + (k-1)F} = \frac{\frac{k-1}{n-k}F}{1+\frac{k-1}{n-k}F}} \sim Beta\left(\frac{k-1}{2},\frac{n-k}{2}\right).
\]

Тогда чтобы посчитать математическое ожидание $R^2$, надо вспомнить, чему равно математическое ожидание для $Beta\left(\frac{k-1}{2},\frac{n-k}{2}\right)$:

\[
E(R^2) = \frac{\frac{k-1}{2}}{\frac{k-1}{2}+\frac{n-k}{2}} = \frac{k-1}{n-1}.
\]

Что нам даёт полученный результат? Математическое ожидание коэффициента детерминации линейно по $k$. То есть даже при включении в модель лишних факторов $R^2$ все равно продолжает линейно расти! Однако мы знаем, что истинная модель --- это регрессия на константу, и матожидание ее коэффициента детерминации должно быть равно нулю. В следующем пункте покажем, что использование скорректированного коэффициента детерминации позволяет преодолеть такое нежелательное свойство коэффициента детерминации как линейный рост по параметру $k$ (число регрессоров).

\item[(b)] Скорректированный коэффициент детерминации имеет вид:
\[
R^2_{adj.} = 1 - \left(1 - R^2\right)\frac{n-1}{n-k}.
\]
Рассчитаем математическое ожидание:
\[
E(R^2_{adj.}) = E\left(1 - \left(1 - R^2\right)\frac{n-1}{n-k}\right) = 1-\frac{n-1}{n-k}+\frac{n-1}{n-k}E(R^2) = 
\]
\[
= 1-\frac{n-1}{n-k}+\frac{n-1}{n-k}\frac{k-1}{n-1} = 0.
\]

Скорректированный $R^2$ помог решить проблему линейного роста по $k$!


\item[(c)] Заметим, что выражение для $R^2$ из пункта (а) можно переписать в следующем виде:
\[
R^2 = \frac{\frac{k-1}{n-k}F}{1+\frac{k-1}{n-k}F}} = \frac{\chi^2_{(k-1)}}{\chi^2_{(n-1)}}.
\]
Тогда рассмотрим
$\lim_{n \to \infty} nR^2 = \lim_{n \to \infty} n\frac{\chi^2_{(k-1)}}{\chi^2_{(n-1)}} =  \lim_{n \to \infty} \frac{\chi^2_{(k-1)}}{\frac{\chi^2_{(n-1)}}{n-1}\frac{n-1}{n}}.$\\
Заметим, что $\lim_{n \to \infty}\frac{n-1}{n} = 1$, а $\lim_{n \to \infty}\frac{\chi^2_{(n-1)}}{n-1\\
} = 1$.\\
Следовательно, $\lim_{n \to \infty} nR^2 = \chi^2_{(k-1)}.$

\end{enumerate}

\end{document}

